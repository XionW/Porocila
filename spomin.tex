\documentclass[10pt,a4paper]{article}
\usepackage{ucs}
\usepackage[utf8x]{inputenc}
\usepackage{amsmath}
\usepackage{amsfonts}
\usepackage{amssymb}
\usepackage[pdftex]{graphicx}

\usepackage[slovene]{babel}
\usepackage{lmodern}
\usepackage[T1]{fontenc}

\author{Vida Groznik, Anže Starič}
\title{Spomin}

\begin{document}
\maketitle
\newpage

\section{Uvod}
Spomin igra pomembno vlogo v našem življenju, saj predstavlja osnovo za učenje, sklepanje in razumevanje. Brez spomina bi dojemali le sedanjost in nebi mogli razumeti vzročnosti, saj bi do trenutka, ko bi videli rezultat neke akcije že pozabili, da smo akcijo izvedli.

Spomin v grobem delimo na eksplicitni (deklarativni) in implicitni (nedeklarativni) spomin. Kadar govorimo preprosto o spominu najpogosteje mislimo na {\bf eksplicitni spomin}. Do eksplicitnih spominov lahko dostopamo zavestno, enostavno jih tvorimo, ravno tako enostavno jih pozabljamo.

{\bf Implicitnega spomina} po drugi strani ne moremo uporabljati zavestno, vendar naučena opravila lahko opravljamo tudi brez zavestnega priklica (vožnja kolesa, prostoročno tipkanje, ...). Za tvorjenje implicitnih spominov potrebujemo več ponavljanja in vaje kot za tvorjenje eksplicitnih spominov, vendar jih zato težje pozabimo.

\section{Eksplicitni spomin}
Kot smo omenili v uvodu, eksplicitne spomine tvorimo in do njih dostopamo zavestno. Spominsko delovanje ločimo na pet procesov: {\it pozornost}, {\it vkodiranje}, {\it shranjevanje}, {\it konsolidacija} in {\it obnavljanje informacij}. {\it Pozornost} je sposobnost osredotočanja na določeno nalogo in igra pomembno vlogo pri izboru informacij, ki vstopajo v spominski proces. Drugi korak je {\it vkodiranje}, ki predstavlja registracijo informacij med procesom učenja. Če lahko informacijo povežemo z že znanimi asociacijami, je postopek vkodiranja bolj učinkovit. Ko je informacija vkodirana je {\it shranjena} v kratkoročni spomin. {\it Konsolidacija} je postopek organizacije kompleksnih informacij, pri katerem se prenesejo v dolgoročni spomin. Pri dostopanju do informacij gre za procese {\it obnavljanja informacij}, poznamo spontan priklic, priklic s pomočjo namiga in prepoznavanje. S priklicom in prepoznavanjem najlažje merimo sposobnost posameznikovega spominskega sistema.

Postopno pozabljanje je običajen proces našega spominskega sistema, kadar pa pride do večje izgube spomina, ki je posledica bolezni ali poškodbe govorimo o amneziji. Poznamo dve vrsti amnezije, {\it retrogradno amnezijo}, pri kateri gre za izgubo spominov pred poškodbo in {\it anterogradno amnezijo}, pri kateri gre za nezmožnost tvorjenja spominov po poškodbi. Bolnik, ki je doživel poškodbo pri 32 letih, katere posledica je bila retrogradna amnezija se tako ne more spomniti dgodkov, ki so se zgodili med njegovim 30 in 32 letom, če pa je posledica anterogradna amnezija, se ne more spomniti ničesar, kar se je zgodilo po njegovem 32 letu.

\subsection{Temporalni reženj}
\begin{figure}[h]
  \centering
    \includegraphics[width=1.0\textwidth]{LokacijaHippocampusa.png}
  \caption{Kje se nahaja hippocampus}
  \label{sHippocampus}
\end{figure}

\subsubsection{Bolnik H.M.}
Pri bolniku H.M so se okrog desetega leta začeli pojavljati epileptični napadi. Z leti so postajali vse hujši, vključevali so tresenje, grizenje jezika in izgubo zavesti. Ker zaradi epilepsije kljub jemanju zdravil ni mogel normalno delati, so mu pri 27 letih med operacijo izrezali dela obeh temporalnih režnjev. Od operacije dalje se pri bolniku epileptični napadi niso več pojavljali.

Odstranitev temporalnih režnjev ni vplivala na zavedanje, inteligenco ali osebnost bolnika H.M., vendar se je pri njem pojavila huda oblika amnezije. Opazni so bili znaki obeh vrst amnezije. Delna retrogradna amnezija se je kazala kot izguba spomina o dogodkih nekaj let pred operacijo, vendar je bolnik še vedno lahko priklical določene spomine iz otroštva. Anterogradna amnezija je bila prisotna v hujši obliki, zaradi nje bolnik ni prepoznal stvari, ki jih je spoznal nekaj minut nazaj. Bolnik si ni zapomnil niti zdravnice, ki je z njim delala vsak dan skoraj 50 let. Pri izvajanju preizkusa delovnega spomina (pomnjenje zaporedja števil) je bolnik dosegel običajen rezultat. V primeru, da je med izvajanjem poizkusa preusmeril pozornost na nekaj drugega, pa je poleg števil pozabil tudi, da si je moral zapomniti števila.

\subsubsection{Vloga temporalnega režnja}
Iz primera bolnika H. M. lahko sklepamo, da po odstranitvi medialnih temporalnih režnjev dolgoročni in delovni spomin nista bila poškodovana, pojavila pa se je okvara sistemov za tvorjenje novih spominov. Pri operaciji so bili odstranjeni del hippocampusa, amygdala ter del korteksa (rhinalni sulcus in parahipocampalni cortex). Če opazujemo povezave v možganih opazimo, da vhod v medialni temporalni reženj prihaja iz asociacijskih področij. Informacije prehajajo preko korteksa do hippocampusa. V asociacijskih področjih se nahajajo močno obdelane informacije različnih modalnosti, kar pomeni, da vhod predstavljajo kompleksne predstavitve. Glavna izhodna povezava je {\it fornix}, ki gre mimo thalamusa in se konča v hypothalamusu.

Hippocampus skupaj z bližnjim korteksom sodeluje pri pretvarjanju informacij, ki prihajajo iz asociacijskih področij. Izgleda tudi, da strukture medialnih temporalnih režnjev sodelujejo pri konsolidaciji spominov. Hkraten pojav retrogradne in anterogradne amnezije pri bolnikih z okvaro medialnih temporalnih struktur namiguje tudi, da se v tem področju nekaj časa hranijo spomini, dokler se dokončno ne konsolidirajo v neokorteksu.

\section{Diencephalon}
\subsection{Bolnik N.A.}
N.A. je bil po poklicu vzdrževalec radarjev v ameriških zračnih silah. Nekega dne je sestavljal model letala, njegov cimer pa je za njegovim hrbtom vadil mečevanje. Ko se je N.A. v nepravem trenutku obrnil, se mu je meč skozi desno nosnico zapičil v možgane. Čez več let so ugotovili, da mu je pri te meč poškodoval le levi del thalamusa, ostali del možganov je bil nepoškodovan.

Po okrevanju se je kognitivna sposobnost vrnila na normalen nivo, vendar pa je imel težave s spominom. Imel je retrogradno amnezijo za obdobje dveh let pred nesrečo in precej hudo obliko anterogradne amnezije. Le-ta ni bila tako huda kot pri bolniku H.M, saj se je lahko spomnil nekaterih dogodkov in obrazov po nesreči, vendar so bili spomini površni. Čeprav je bila amnezija bolnika N.A. blažja, je po znakih zelo podobna amneziji bolnika H.M, delovni in dolgoročni spomin nista bila prizadeta, imel pa je težave s tvorjenjem novih spominov in retrogradno amnezijo, iz česar sklepamo, da tudi thalamus igra pomembno vlogo pri konsolidaciji spominov.



\section{Implicitni spomin}
\subsection{Proceduralni spomin}
Za proceduralni spomin bi v grobem lahko rekli, da je to spomin za opravljanje različnih nalog. Ko se pojavi potreba po opravljanju neke naloge, se avtomatsko prikliče in uporabi proceduralni spomin, ki vsebuje tako kognitivne kot tudi motorične spretnosti. Proceduralni spomin je vrsta dolgoročnega spomina, natančneje implicitnega spomina.

\subsubsection{Anatomska struktura}
\textbf{Striatum in bazalni gangliji}\\
Dorzolateralni striatum je povezan z učenjem navad in je glavno nevronsko celično jedro, ki je povezano s proceduralnim spominom. Povezovanje živčnih vlaken pomaga pri regulaciji aktivnosti v bazalnih ganglijih. Iz striatuma potujeta dve vzporedni poti za procesiranje informacij, ki delujate ena nasproti drugi pri kontroliranju gibanja in omogočata povezovanje z drugimi potrebnimi funkcionalnimi strukturami. Prva pot je neposredna, druga posredna in skupaj delujeta na tak način, da omogočata funkcionalne nevronske povratne zanke. Veliko povratnih povezav povezuje različne dele možganov s striatumom. Glavna povratna povezava, ki je povezana z motoričnimi spretnostmi, se imenuje korteks-bazalni gangliji-talamus-korteks zanka.

Dosedanje razumevanje možganske anatomije in fiziologije kažejo, da je stratumska zgradba tista, ki omogoča komunikacijo bazalnih ganglijev s strukturami in funkcionalno delovanje v procesiranju proceduralnega spomina. 
\\
\\
\textbf{Mali možgani}\\
Mali možgani so znani po tem, da igrajo ključno vlogo pri popravljanju gibanja in po piljenju proceduralnih motoričnih sposobnosti kot so slikanje, igranje instrumenta, in celo igranje golfa. Poškodbe tega dela lahko preprečijo ustrezno učenje motoričnih sposobnosti in s tem povezane raziskave so pokazale, da ima ta del vlogo pri avtomatiziranju nezavednih procesov pri učenju proceduralnih spretnosti.
\\
\\ 
\textbf{Limbični sistem}\\
Limbični sistem je skupina edinstvenih področij možganov, ki sodelujejo v številnih medsebojnih procesih, ki vključujejo čustva, motivacijo, učenje in spomin. TREnutna razmišljanja kažejo, da limbični sistem deli anatomijo z delom neostriatuma, ki je zaslužen za velike naloge nadzorovanja proceduralnega spomina. Ta vitalni del možganov najdemo na striatumovi zadnji meji in je bila šele pred kratkim povezana s spominom, sedaj pa jo imenujemo mejna delitev območja (MrD). Sledenje aktivacijam možganskih delov, ki sodelujejo med delovanjem proceduralnega spomina je možno zaradi proteinov v membrani limbičnega sistema.

\subsubsection{Testi}
Za testiranje delovanja spomina se uporablja več različnih testov.
\\
\\
\textbf{Zasledovanje premikanja}\\
Za preučevanje motoričnega učenja oseba z miškinim kurzorjem sledi premikajočemu objektu na ekranu. Tako izmerimo proceduralni spomin in pokažemo, koliko so izpiljeni gibi testirane osebe. Rezultati so prikazani kot razmerje med tem, koliko časa je oseba imela kurzor na objektu in koliko časa ne. Osebe z amnezijo ne kažejo oslabitve pri tem motoričnem testu pri nadaljnjih ponovitvah poizkusa. Vendar pa na to sposobnost vplivajo primankljaj spanca in uporaba drog.
\\
\\
\textbf{Merjenje reakcijskega časa}\\
Ta naloga vključuje, da sodelujoči ohranijo in se učijo proceduralnih sposobnosti, ki ocenjuje spomin za proceduralne motorične sposobnosti. Te sposobnosti se merijo z opazovanjem hitrosti in točnosti učenja ter ohranjanja novih sposobnosti. Reakcijski čas je čas, ki ga potrebuje udeleženec, da se odzove na dano iztočnico. Udeleženci z Alzheimerjevo boleznijo in amnezijo lahko dolgo časa hranijo naučeno sposobnost in lahko kasneje učinkovito opravijo takšno nalogo.
\\
\\
\textbf{Zrcalno sledenje}\\
Ta naloga se osredotoča na natančnejše povezovanje čutov, saj je to vizualno motorični test, kjer se morajo udeleženci naučiti nove motorične sposobnosti, ki vključuje koordinacijo rok in oči. Udeleženci z amnezijo so se sposobni naučiti in zapomniti to nalogo, kar nakazuje na proceduralni spomin. Risanje slike je namreč delo proceduralnega spomina; ko enkrat ugotovimo kako narisati podobo v zrcalu, imamo naslednjič s tem manj težav. Posamezniki z Alzheimerjevo boleznijo se ne morejo spomniti veščin, ki so jih pridobili v tej nalogi, vendar pa ne glede na uspešnost pridobijo proceduralno sposobnost.
\\
\\
\textbf{Naloga z napovedovanjem vremena}\\
Ta naloga uporablja eksperimentalno analizo napovedovanja vremena. To je naloga učenja verjetnosti v kateri mora udeleženec navesti, kakšno strategijo se uporablja za reševanje te naloge. Ta je orientirana kognitivno in se jo nauči na proceduralni način. Izvede se jo tako, da udeleženci dobijo set kartic z različnimi oblikami na podlagi katerih morajo napovedati izzid. Po podani napovedi udeleženci dobijo povratno informacijo in naredijo klasifikacijo na podlagi teh informacij. Udeleženci z amnezijo se med samim testiranjem te naloge naučijo, vendar se kasneje, ob ponovnem testiranju odrežejo precej slabše.


\subsubsection{Motnje}
Alzheimerjeva bolezen in demenca\\
Sindrom Gilles de la tourete\\
Virus HIV\\
Huntingtonova bolezen\\
Obsesivna kompulzivna motnja (OCD)\\
Parkinsonova bolezen\\
Šizofrenija

\subsubsection{Vpliv drog}
Alkohol\\
Kokain\\
Psihostimulanti

\section{}
\end{document}