\documentclass[10pt,a4paper]{article}
\usepackage{ucs}
\usepackage[utf8x]{inputenc}
\usepackage{amsmath}
\usepackage{amsfonts}
\usepackage{amssymb}

\usepackage[slovene]{babel}
\usepackage{lmodern}
\usepackage[T1]{fontenc}

\author{Vida Groznik, Anže Starič}
\title{Spomin}

\begin{document}
\maketitle
\newpage

\section{Uvod}
Spomin igra pomembno vlogo v našem življenju, saj predstavlja osnovo za učenje, sklepanje in razumevanje. Brez spomina bi dojemali le sedanjost in nebi mogli razumeti vzročnosti, saj bi do trenutka, ko bi videli rezultat neke akcije že pozabili, da smo akcijo izvedli.

Spomin v grobem delimo na eksplicitni (deklarativni) in implicitni (nedeklarativni) spomin. Kadar govorimo preprosto o spominu najpogosteje mislimo na {\bf eksplicitni spomin}. Do eksplicitnih spominov lahko dostopamo zavestno, enostavno jih tvorimo, ravno tako enostavno jih pozabljamo.

{\bf Implicitnega spomina} po drugi strani ne moremo uuporabljati zavestno, vendar naučena opravila lahko opravljamo tudi brez zavestnega priklica (vožnja kolesa, prostoročno tipkanje, ...). Za tvorjenje implicitnih spominov potrebujemo več ponavljanja in vaje kot za tvorjenje eksplicitnih spominov, vendar jih zato težje pozabimo.

\section{Eksplicitni spomin}
\subsection{Delovni spomin}
Anze was here!

\subsection{Kratkoročni spomin}

\subsection{Dolgoročni spomin}

\section{Implicitni spomin}
\subsection{Proceduralni spomin}


\section{}
\end{document}