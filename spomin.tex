\documentclass[10pt,a4paper]{article}
\usepackage{ucs}
\usepackage[utf8x]{inputenc}
\usepackage{amsmath}
\usepackage{amsfonts}
\usepackage{amssymb}
\usepackage[pdftex]{graphicx}

\usepackage[slovene]{babel}
\usepackage{lmodern}
\usepackage[T1]{fontenc}

\author{Vida Groznik, Anže Starič}
\title{Spomin}

\begin{document}
\maketitle
\newpage

\section{Uvod}
Spomin igra pomembno vlogo v našem življenju, saj predstavlja osnovo za učenje, sklepanje in razumevanje. Brez spomina bi dojemali le sedanjost in nebi mogli razumeti vzročnosti, saj bi do trenutka, ko bi videli rezultat neke akcije že pozabili, da smo akcijo izvedli.

Spomin v grobem delimo na eksplicitni (deklarativni) in implicitni (nedeklarativni) spomin. Kadar govorimo preprosto o spominu najpogosteje mislimo na {\bf eksplicitni spomin}. Do eksplicitnih spominov lahko dostopamo zavestno, enostavno jih tvorimo, ravno tako enostavno jih pozabljamo.

{\bf Implicitnega spomina} po drugi strani ne moremo uuporabljati zavestno, vendar naučena opravila lahko opravljamo tudi brez zavestnega priklica (vožnja kolesa, prostoročno tipkanje, ...). Za tvorjenje implicitnih spominov potrebujemo več ponavljanja in vaje kot za tvorjenje eksplicitnih spominov, vendar jih zato težje pozabimo.

\section{Eksplicitni spomin}
Delovanje eksplicitnega spomina pri človeku povezujemo predvsem z dvema deloma možganov. Prvi je hipokampus, ki se nahaja v medialnem delu temporalnega režnja, drugi pa diencephalon. Spominsko funkcijo hipokampusa si bomo ogledali na primeru bolnika H.M.

\subsection{Hipokampus}
O pomembni vlogi hipokampusa pri tvorbi in hrambi deklarativnih spominov nam govori amnezija, ki nastane kot rezultat odstranitve obeh temporalnih režnjev. Posledice odstranitve si oglejmo na primeru bolnika H.M.

\subsubsection{Bolnik H.M.}
Pri bolniku H.M so se okrog desetega leta začeli pojavljati epileptični napadi. Z leti so postajali vse hujši, vključevali so tresenje, grizenje jezika in izgubo zavesti. Ker zaradi epilepsije kljub jemanju zdravil ni mogel normalno delati, so mu pri 27 letih med operacijo izrezali 8 cm medialnih delov obeh temporalnih režnjev, kar vključuje del korteksa, amygdalo in dve tretjini hipokampusa. Po operaciji so se epileptični napadi prenehali.

Odstranitev temporalnih režnjev ni vplivala na zavedanje, inteligenco ali osebnost bolnika H.M., vendar se je pri njem pojavila huda oblika amnezija. Amnezija je delno retrogradna, saj je izgubil spomin o dogodkih nekaj let pred operacijo, hujša pa je njena anterogradna oblika. Bolnik lahko prikliče nekatere dogodke iz otroštva, vendar pa se ne more spomniti nekoga, ki ga je videl pred nekaj minutami. Bolnik si ni zapomnil niti zdravnice, katero je skoraj 50 let videval vsak dan. Če na njem izvajamo poskus, pri katerem si mora bolnik zapomniti določeno število, nato pa njegovo pozornost preusmerimo drugam, bolnik ne le pozabi število, temveč pozabi tudi, da smo mu kadarkoli dali nalogo, da si število zapomni.

Bolnik ima ohranjen dolgoročni spomin, saj se spomni nekaterih dogodkov iz otroštva. Ravno tako je ohranjen delovni spomin, saj si lahko s ponavljanjem zapomni seznam števil, vendar takoj, ko pozornost preusmeri nekam drugam seznam števil pozabi. Na podlagi bolnika H.M. lahko sklepamo, da hipokampus igra pomembno vlogo pri tvorbi in hranjenju kratkotrajnega spomina.

\subsubsection{Vloga medialnega temporalnega režnja}
Poleg hippocampusa se na medialnem temporalnem režnju nahajata še rhinalni sulcus in parahippocampalni cortex. Medialni temporalni reženj informacije prejema iz asociacijskega področja, ki vsebuje močno obdelane informaciej iz vseh modalnosti. Vhod torej predstavljajo kompleksne predstavitve vhodov s čutil, pri katerih se lahko mešajo modalnosti. Glavna izhodna povezava je {\bf fornix}, ki potuje mimo talamusa in se konča v hipotalamusu.

\begin{figure}[h]
  \centering
    \includegraphics[width=1.0\textwidth]{LokacijaHippocampusa.png}
  \caption{Kje se nahaja hippocampus}
  \label{sHippocampus}
\end{figure}

\section{Eksplicitni spomin}
\subsection{Delovni spomin}

\subsection{Kratkoročni spomin}

\subsection{Dolgoročni spomin}

\section{Implicitni spomin}
\subsection{Proceduralni spomin}


\section{}
\end{document}